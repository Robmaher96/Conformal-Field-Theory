\documentclass{article}
\usepackage[utf8]{inputenc}

\title{Conformal Field Theory}
\author{Rob Maher}
\date{October 2018}
\usepackage{amsthm}
\usepackage{amssymb}
\usepackage{amsmath}
\usepackage{braket}


\newcommand{\R}{\mathbb{R}}
\newcommand{\Q}{\mathbb{Q}}
\newcommand{\Z}{\mathbb{Z}}
\newcommand{\C}{\mathbb{C}}
\newcommand{\N}{\mathbb{N}}


\begin{document}


\maketitle

\section{The Conformal Group and Algebra}

Given a differentiable map $\varphi: U \rightarrow V$ where $U$ and $V$ are open subsets of manifolds $M$ and $M'$, then $\varphi$ is called conformal if the metric tensor satisfies $$\varphi^*g'=\Lambda g$$ for $\Lambda=\Lambda(x)$ some positive function. In what we follows, we assume that $M=M'$ and that the metric is some diagonal matrix with 1 or -1 along the diagonal. In other words $g_{\mu\nu}=\eta_{\mu\nu}$.

The conformal group is the group consisting of globally defined, invertible, finite conformal transformations. The conformal algebra is the Lie algebra corresponding to this group. 

Consider an infinitesimal transformation $$x'^{\rho}=x^{\rho}+\epsilon^{\rho}(x)+\cdots$$ Substituting this into the above definition yields the condition for infinitesimal conformal transformations: 

$$\partial_{\mu}\epsilon_{\nu}+\partial_{\nu}\epsilon_{\mu}=\frac{2}{d}\partial \cdot \epsilon \eta_{\mu\nu}$$ where $d=\text{dim} M$.

Solving this equation for $\epsilon_{\mu}$ for $d \geq 3$ gives that the conformal algebra for $\R^{p,q}$ is $\mathfrak{so}(p+1,q+1)$ and it is important to note here that this is a finite dimensional Lie algebra.

For $d=2$, we have that $\epsilon$ satisfies the Cauchy-Riemann equations. We introduce the natural complex coordinates $$ z=x^0+ix^1 \quad \overline{z}=\overline{x}^0-i\overline{x}^1 \quad \epsilon=\epsilon^0+i\epsilon^1 \quad \overline{\epsilon}=\overline{\epsilon}^0-i\overline{\epsilon}^1.$$  The corresponding Lie algebra is then generated by $$l_n=-z^{n+1}\frac{d}{dz} \text{ and } \overline{l}_n=-\overline{z}^{n+1}\frac{d}{d\overline{z}}.$$

The only globally defined generators are $l_{-1},l_0$ and $l_1$. These generate the conformal algebra, the so-called Witt algebra. The corresponding conformal group is the group of M\:obius transformations, $$z \mapsto \frac{az+b}{cz+d}.$$ 

We then define the Virasoro algebra to be the central extension of the Witt algebra given by the commutator $$[L_m,L_n]=(m-n)L_{m+n}+\frac{c}{12}(m^3-m)\delta_{m+n,0}$$
where $c$ is called the central charge. 

\section{Fields}
A field $\phi$ that only depends on $z$ are called chiral. If a field only depends on $\overline{z}$ then it is anti-chiral. 

Given any conformal transformation $f(z)$, if a field transforms as $$\phi'(z,\overline{z})=\Big(\frac{\partial f}{\partial z}\Big)^h\Big(\frac{\partial f}{\partial \overline{z}}\Big)^{\overline{h}}\phi(f(z),f(\overline{z}))$$ then $\phi$ is called a primary field of dimensions $(h,\overline{h})$. If this only holds for globally defined transformations, then $\phi$ is called quasi-primary. 

The energy momentum tensor diagonalises into a chiral and anti chiral field $$T_{zz}(z,\overline{z})=T(z) \text{ and } T_{\overline{zz}}(z,\overline{z})=T(\overline{z}).$$ Note that $T(z)$ is not in general a primary field, but it is a quasi-primary field of dimension $(2,0)$. 

We compactify the spatial coordinate $x^1$ onto a unit circle and so obtain an infinite cylinder. We map this cylinder to the complex plane using the transformation $$z=e^w.$$
Any field can expanded as a Laurent series $$\phi(z,\overline{z})=\sum_{n, \overline{m} \in \Z}z^{-n-h}\,\overline{z}^{-\overline{m}-\overline{h}}\phi_{n,\overline{m}}$$ where $\phi_{n,\overline{m}}$ are thought of as operators upon quantisation. 
An asymptotic in state, $x^0 \rightarrow -\infty$ in the transformation $z=e^w$ becomes $z,\overline{z} \rightarrow 0$. The in-state corresponding to the primary field is thus defined to be $$\ket{\phi}=\lim_{z,\overline{z} \rightarrow 0}\phi(z,\overline{z})\ket{0}.$$ However, due to singularty issues at $z=0$, the condition $$\phi_{n,\overline{m}}\ket{0}=0$$ for $n>-h$ and $\overline{m}>\overline{h}$ must be enforced. Therefore, the in-state reduces to $$\ket{\phi}=\phi_{-h,\overline{h}}\ket{0}.$$

\section{The Operator Product Expansion and Normal Ordering}

It can be shown that an equivalent definition of a primary field of conformal dimensions $(h,\overline{h})$ is a field $\phi(z,\overline{z})$ that has operator product expansion 

\begin{align*}
T(z)\phi(w,\overline{w})=\frac{h}{(z-w)^2}\phi(w,\overline{w})+\frac{\partial_w\phi(w,\overline{w})}{z-w}+\cdots\\
\overline{T}(\overline{z})\phi(w,\overline{w})=\frac{\overline{h}}{(\overline{z}-\overline{w})^2}\phi(w,\overline{w})+\frac{\partial_{\overline{w}}\phi(w,\overline{w})}{\overline{z}-\overline{w}}+\cdots
\end{align*}

Here we have implicitly assumed radial ordering, that is $|z| > |w|$. 

The operator product expansion of the chiral energy momentum tensor with itself reads
$$T(z)T(w)=\frac{c/2}{(z-w)^4}+\frac{2}{(z-w)^2}T(w)+\frac{\partial_wT(w)}{z-w}+\cdots$$

In proving this statement, one finds that the modes for the energy momentum tensor given by $$T(z)=\sum_{n \in \Z}z^{-n-2}L_n$$ are precisely the generators of the Virasoro algebra. One also discovers that the OPE of a primary field and energy momentum tensor is equivalent to $$[L_m,\phi_m]=((h-1)m-n)\phi_{m+n}$$

Furthermore, the OPE can be used to derive the Conformal Ward Identity:

$$\langle T(z)\phi_1(w,\overline{w})\dots\phi_N(w,\overline{w})\rangle=\sum_{i=1}^N\Big(\frac{h_i}{(z-w_i)^2}+\frac{\partial_{w_i}}{z-w_i} \Big)\langle \phi_1(w,\overline{w})\dots\phi_N(w,\overline{w})\rangle$$

In quantum field theory, one desires to have a field which contains normal ordered products. This means creation operators should be to the left of annihilation operators. Recalling that $\phi_{n,\overline{m}}\ket{0}=0$ for $n>-h$ and $\overline{m}>\overline{h}$, we see that these modes are annihilation operators. Correspondingly, $\phi_{n,\overline{m}}$ for $n\leq-h$ and $\overline{m}\leq\overline{h}$ are creation operators. 

The holomorphic part of an OPE of two primary fields can be written such that it only contains normal ordered products, that is $$\phi(z)\chi(w)=\cdots+\sum_{n=0}^{\infty}\frac{(z-w)^n}{n!}N(\chi\partial^n\phi)(w)$$

The invariance of the two and three point functions under $\text{SL}(2,\C)/\Z_2$ means that we can find the form of these correlators up to a constant. These constants determine the general form of an OPE. For non chiral fields, an OPE is additionally constrained by crossing symmetry. The four point function can be expanded in terms of conformal blocks as $$G(\textbf{z},\overline{\textbf{z}})=\sum_pC^p_{ij}C^p_{lm}\mathcal{F}_{ij}^{lm}(p|x)\overline{\mathcal{F}}^{lm}_{ij}(p|\overline{x})$$ where $x$ is the crossing ratio. 

\section{The Hilbert Space of States}

For each state $\ket{\phi}$ in the Verma module, given by $$\{L_{k_1}\dots L_{k_n}\ket{0}:\, k_i \leq 2\}$$ there exists a field $F(z)$ with the property that $\lim_{z \rightarrow 0}\ket{0}=\ket{\phi}$. Here we have implicitly assumed the existence of the vacuum state $\ket{0}$.  

Using the definition of a primary field one can show $$L_n\ket{\phi}:=L_n\ket{h}=L_n\phi_{-h}\ket{0}=0 \text{ and } L_0\ket{h}=h\ket{h}$$ for $n>0$. This means that $\ket{h}$ is a highest weight state and new states are created by combinations of $L_n$ for $n<0$. This forms the representation of the Virasoro algebra. These states correspond to descendent fields $\phi, \partial\phi, N(T\phi), \dots$. The set of all descendent fields $[\phi]$ is called the conformal family. 

The set of states of zero norm in a Verma module generate and submodule. Thus, if a Verma module contains null states, the representation is reducible. The constraint on the conformal dimension in a reducible representation is determined by the vanishing of the Kac determinant. Theories which allow states if negative norm are called minimal models. 

For unitary representations, states of negative norm are excluded and one can show that only a discrete set of values of the central charge are allowed. It can be proven further that this happens only for rational values of $c$ and such theories are called Rational Conformal Field Theories.  

For unitary minimal models of the Virasoro algebra, we have the so-called fusion rules.

\begin{align*}
    [\phi_{(m,n)}] \times [\phi_{(p,q)}]=\sum_{\substack{k=1+|m-p|\\k+m+p \text{ odd}}}^{m+p+1}\sum_{\substack{l=1+|n-q|\\k+n+q \text{ odd}}}^{n+q+1}[\phi_{k,l}]
\end{align*}

The definition of the fusion algebra is $$[\phi_i] \times [\phi_j]=\sum_kN^k_{ij}[\phi_k].$$ The algebra is commutative so $N^k_{ij}=N^k_{ji}$. The algebra is also associative and so defining the matrix $(\mathcal{N}_i)=N^k_{ij}$ we obtain $$\mathcal{N}_i\mathcal{N}_k=\mathcal{N}_j\mathcal{N}_i$$


\begin{displaymath}
	E = mc^{2}
\end{displaymath}

Does this work?

\end{document}
